\documentclass[a4paper,12pt]{book}
\usepackage[utf8]{inputenc}
\usepackage{textcomp} 

%\pdfmapfile{+universalis.map}
%\usepackage[sfdefault]{universalis}
%\usepackage[scaled]{helvet}
\renewcommand\familydefault{\sfdefault}

%\usepackage[math]{kurier}
%\usepackage[T1]{fontenc}


\usepackage{fancyhdr}
\pagestyle{fancy}
\usepackage[]{qrcode}

\usepackage{infoheft}





%http://stackoverflow.com/questions/877597/how-do-you-change-the-document-font-in-latex





\begin{document}
%\maketitle



\MyHeadingtext{Stammverbang | Tauchen}
\begin{body}


 
\section*{Einsatztaucherausbildung ET1/ET2}
\subsection*{Modulausbildung zum Einsatztaucher Stufe 2}
Der DLRG Landesverband Schleswig-Holstein bietet in 2015/2016
eine zentrale Ausbildung zum Einsatztaucher Stufe 2 an. Die Ausbildung hat bereits begonnen und das erste Modul hat bereits stattgefunden. Die gesamte Ausbildung 
inkl. Prüfungswochenende besteht aus 5 Modulen die zwischen Okt. 2015 und Okt.
2016 angeboten werden. Ausbildungsumfang: 120 UE. Die
Ausbildung findet immer im Landeszentrum in Eckernförde statt.
Jedes Modul geht von Freitag bis Sonntag und kostet 70,00 €. 
Die weiteren Module in 2016 sind geplant für: 

\begin{itemize}
\item Modul 2: 11. - 13. März 2016 
\item Modul 3: 22. - 24. April 2016 
\item Modul 4: 20. - 22. Mai 2016 
\item Modul 5: 9.-11 September 2016
\end{itemize}
%Einsatztaucher der Stufe 1 können 2016 die Module 4 und 5 besuchen um ET2  zu werden
\end{body} 

\begin{targetaudiencediv}
\subsubsection*{Zielgruppe}
DLRG-Einsatztaucheranwärter
\end{targetaudiencediv} 

\begin{requirementsdiv}
\subsubsection*{Voraussetzungen}
Voraussetzungen für die Teilnahme gemäß Prüfungsordnung Tauchen 613: 
\begin{itemize}
\item Mitglied DLRG
%\item 18 Jahre
%\item RS Silber mit EH-Kurs, nicht älter als 2 Jahre
%\item Deutsches Schnorcheltauchabzeichen (161) – entfällt, da Sporttauchschein gefordert!
%\item Sprechfunkunterweisung – kann ggf. nachgeliefert werden
\end{itemize}
\end{requirementsdiv}
\par
\begin{costdiv}
\begin{tabular}{@{}lll}
\begin{minipage}[t]{0.4\textwidth}
\subsubsection*{Kosten} 
DLRG Mitglieder SH : 70 € \\
externe Teilnehmende: 140 € 
\end{minipage} &
\begin{minipage} [t]{0.25\textwidth}
\subsubsection*{AP-Fortbildung} 
Nein 

\end{minipage} &
\begin{minipage} [t]{0.3\textwidth}
\subsubsection*{JuLeiCa-Fortbildung} 
Nein
\end{minipage} 
\end{tabular} 
\end{costdiv} 
%\vspace{12pt}
\marginpar{AP-Fortbildung}

\begin{detailsdiv}
\begin{tabular}[t]{@{}ll}
\begin{minipage}[t]{0.4\textwidth}

%   \begin{pspicture}(1in,1in)
%     \psbarcode{Hier steht der Link zum Lehrgang im Internet}{}{qrcode}
%   \end{pspicture}
%\hline
\subsubsection*{Online-Anmeldung} 
\qrcode[hyperlink,height=35mm]{http://www.sh.dlrg.de/hier_ist_der_link_zum_lehrgang} 
%\hline
\end{minipage} &
\begin{minipage}[t][30mm][t]{0.6\textwidth}
\vspace{12pt}

\dateEntry{24.03 - 29.03.2016}

\seminardetail{Lehrgangs-Nr}{T1-23-2016}
\seminardetail{Meldeschluss}{08.04.2016}
\seminardetail{Leitung}{Frank Hertlein}
\seminardetail{Ort}{Eckernförde}
\seminardetail{Zeit}{Freitag, 20.05.2016 18:00 Uhr bis}
\seminardetail{}{Sonntag, 22.05.2016 18:00 Uhr}

\end{minipage} 
\end{tabular} 
\end{detailsdiv} 



\end{document}
