

\renewcommand\myheadingtext{Gesamtverband | Organisatorisches}

\section*{Teilnahmebedingungen}
\subsection*{Frühes Anmelden sichert Plätze!}

Im Bereich der Techniklehrgänge werden 50\% der Plätze nach Eingang der Anmeldungen vergeben. Windhund"=Prinzip!
Zur einfachen Abwicklung unserer Seminare benutzt bitte die Online"=Anmeldung unter: \textbf{sh.dlrg.de/seminare}

\paragraph{Teilnahmebestätigung}
Nach Ablauf des Anmeldeschlusses bekommt jede/r Teilnehmer/in
per E"=Mail eine Teilnahmebestätigung mit weiteren Informationen
zur Veranstaltung (Anfahrtsweg, zeitlicher Ablauf, Themen etc.). Nur
der Erhalt dieser Teilnahmebestätigung gilt als Anspruch auf die Seminarteilnahme.

\paragraph{An-, Ab- und Ummeldungen}
Eine kostenfreie Abmeldung ist bis zum Anmeldeschluss möglich.
Danach werden die entstandenen Kosten in Höhe des Teilnahmebetrages in Rechnung gestellt. An"=, Ab"= und Ummeldungen müssen
schriftlich erfolgen.
Die Anmeldungen müssen über die örtliche Gliederung online bestätigt werden. Um"= und Ersatzmeldungen zu diesen Seminaren
sind nur mit Bestätigung der Gliederung möglich.
\marginpar{\includegraphics[width=1\marginparwidth]{./Logo-B-Schwarz.png}}
\marginpar{\includegraphics[width=1\marginparwidth]{./dlrg-jugend-sw-vektor}}

\paragraph{Teilnahmegebühr}
Die Abrechnung erfolgt durch das SEPA"=Basis"=Lastschriftmandat.
Hierzu muss unbedingt die Berechtigung zum Bankeinzug für Seminaranmeldungen vorliegen. Wir bitten vorab zu prüfen, ob die
Gliederung oder der/die Teilnehmer/in die Gebühr trägt. Bei „Technik”"=Seminaren zahlt immer die Gliederung. Bei Anmeldungen
ohne die Berechtigung zum Bankeinzug werden pro Teilnehmer/in
10 Euro Bearbeitungsgebühr zusätzlich berechnet.
Teilnehmer/innen aus dem Landesverband Schleswig"=Holstein zahlen den Mitgliedspreis, 
alle anderen den Preis für externe Teilnehmende. Wenn nicht anders angegeben, beträgt der Preis für externe
Teilnehmende das Doppelte des Mitgliedspreises.

\newpage
\section*{Teilnahmebedingungen}
\subsection*{Frühes Anmelden sichert Plätze!}


\textbf{Minderjährige Teilnehmer/innen}\newline
Minderjährige Teilnehmer/innen müssen nach Online"=Anmeldung
die schriftliche Einverständniserklärung (Formular kommt per Email)
der Erziehungsberechtigten vorlegen.
Diese Zustimmung kann jederzeit schriftlich mit Wirkung für die Zukunft widerrufen werden.\newline

\textbf{Rechte an den auf den Seminaren gemachten Bildern}\newline
Der/die Die Teilnehmer/in erklärt sich damit einverstanden, dass alle
Fotos, Filme und Tonaufnahmen, die von ihm/ihr gemacht werden,
im Rahmen der gesetzlichen. Vorschriften für 
Dokumentations"=, Informations"= und Werbezwecke der DLRG genutzt und veröffentlicht
werden dürfen (z.B. im Radio, Fernsehen, Internet und in den Printmedien).
\newline
DLRG im Internet: http://www.sh.dlrg.de
\newline
\marginpar{\includegraphics[width=1\marginparwidth]{./Logo-B-Schwarz.png}}
\marginpar{\includegraphics[width=1\marginparwidth]{./dlrg-jugend-sw-vektor}}

\textbf{Ansprechpersonen:}\newline
\textbf{Bei Fragen zu An"=, Ab"= und Ummeldungen}\newline
Sanja Seemann (Lehrgangsverwaltung)\newline
Tel.: (04351) 71 77 15\newline
EMail: seminare@sh.dlrg.de\newline

\textbf{Bei inhaltlichen Rückfragen zu Technik"=Seminaren}\newline
Frank Hertlein (Technischer Leiter)\newline
Tel.: (0171) 699 87 31\newline
EMail: frank.hertlein@sh.dlrg.de\newline

\textbf{Bei inhaltlichen Rückfragen zu Jugend"=Seminaren\newline
und Juleica (Jugendleiter/in Card)}
Madeleine Brandt (Bildungsreferentin)\newline
Tel.: (04351) 7177 19\newline
EMail: bildungsreferentin@sh.dlrg-jugend.de\newline

\textbf{Bei technischen Problemen}\newline
Support\newline
EMail: support@sh.dlrg.de\newline


\newpage